\newpage
% Roman numerals for page counter
\renewcommand{\thepage}{\roman{page}}

\begin{changemargin}{1cm}{1cm}

\begin{Abstract}
\addcontentsline{toc}{chapter}{Abstract}
\thispagestyle{plain}
\vspace{1cm}
The present document investigates the cavitation phenomenon
occurring in hydrofoils by means of a multiphase solver available 
in the open-source computational fluid dynamic (CFD) software 
OpenFOAM\textsuperscript{\textregistered}. The current study uses 
the Reynolds Average Navier-Stokes methods to numerically 
simulate sheet/cloud cavitation around a NACA0012 hydrofoil
with a fixed cavitation number $\sigma$=0.8. 
The turbulence model k-$\omega$ sst is applied in this thesis to limit
the computational effort. However, multiple refined grids are accounted 
to compare the obtained solutions. Simulation results including cavitation 
form, lift, and drag coefficients related to k-$\omega$ sst turbulence models 
are studied and compared with experimental results. Non-cavitating conditions 
are experienced for angles of attack ranging from $3.2^{\circ}$ to $3.8^{\circ}$; 
in that range the k-$\omega$sst turbulence model performs well. When the angle of 
attack ranges from $4.1^{\circ}$ to $5.0^{\circ}$, the k-$\omega$ sst model fails
to predict cavitation shedding. When the angle of attack is between $5.7^{\circ}$
and $8^{\circ}$, the k-$\omega$ sst model can reproduce cavitation shedding, 
but the numerical result is slightly inaccurate compared to experimental data.
As a result, in the current analysis the k-$\omega$ sst model is unable to 
provide very accurate values of forces coefficient when the angle of attack 
is further increased. Such result is in accordance with the literature chosen 
as basis for this work. An advanced turbulence model is hence recommended 
to guarantee better results.


\end{Abstract}
\end{changemargin}
