\chapter*{Introduction}
\label{chap:introduction}
\renewcommand{\thepage}{\arabic{page}}
The cavitation involves a very complex two-phase flow phenomenon. It can have 
multiple  negative impacts on the hydraulics machines such as through vibration due to pressure fluctuation, high
acoustic emission, solid surface erosion, and drop down in mechanical
performance. Therefore it is necessary to investigate the physical
mechanism for controlling cavitation in engineering applications. The
primary goal is to simulate two-dimensional unsteady cloud shedding in
cavitation and the re-entrant jet of closed type partial cavity over a
NACA0012 hydrofoil at a different angle of attack and with constant
cavitation number. The cloud shedding phenomenon is reproduced by using 
the $k- {\omega}$ sst turbulence method and homogeneous multiphase flow model. The
physical cavitation model used in this study was developed by Schneer
and Sauer called the Schnerr Saucer model. This simulation also shows the
limitation of using the turbulent model at a higher angle of
attack. A multiphase flow setup is used to validate the result by
running the solver InterFoam in OpenFOAM source code.\\ In
reference paper \cite{ZHANG2017} the work was performed on a 3D Clark-Y
hydrofoil by using an improved filter-based turbulence model based on
density correction method to control the overprediction of turbulence viscosity. On the other hand without using
the filter-based model the overprediction of turbulent viscosity which
causes the re-entrant to lose momentum. As a result, unable to cut the
cavity interface and no cloud shedding. The other influencing parameter is the maximum density ratio which could affect the mass transfer rate between the liquid and vapor. With 
the increase of maximum density ratio, the process of phase transition is promoted and the mass transfer rate between the liquid and the vapor phase sees a significant increase in cavity length,
cavity and thickness. Thus the mass transfer from liquid to vapor at a higher maximum density ratio can produce a large vapor cavity volume. The basic two-equation turbulence
model cannot be used for simulation of large-scale cavitation shedding because of overprediction of turbulent viscosity
. So from this paper, it is clear
that the basic turbulence two-equation model would not be able to
predict better results near to the hydrofoil where viscous flow is
dominant but show good results away from the foil within the yplus
outer region where the inertia of flow is dominant. Simulation was performed in the ANSYS.

In reference,  paper \cite{JI2015, JI2017} the simulation was performed on
NACA66 hydrofoil and this work explain the 3D effect in the cloud
shedding through cavitation vortex
interaction. They compare LES Wall Adapting local eddy -viscosity
model over the LES WALE model which has a better ability to reproduce the laminar to
turbulent transition in comparison with LES Smagorinsky. The cavitation
vortex interaction includes vortex stretching, vortex dilation(due
to volumetric expansion/contraction) baroclinic torque term(due to
misaligned pressure gradient and density gradient). They concluded,
as the cavitation grows, there is strong vortex cavitation interaction
in the shedding vapor cloud, and the vortex stretching, dilation term is
the primary function for the transition from 2D to 3D. And the transition
from 3D to 2D after the shed vapor cloud collapses downstream the attached cavity shrinks. During this process the attached cavitation and boundary layer becomes
thin. This process of transition will continue after the regular interval time. While comparing experimental pressure fluctuation with numerical, there is a shift in the result 
 due to the streamwise vortex (one of the reasons) in the real flow setup
which could not able to add to the simulation setup. So from this
paper, it is understood that cavitation shedding condition is required for producing cloud
shedding in the simulation along with the effect of 3D, and its
contribution has an impact on the numerical result. Simulation was performed in ANSYS.  

In reference, \cite{Hidalgo2014,Hai2020,Hidalgo2014-2} the study work was validated
using OpenFOAM in NACA66. In this work, there is a comparison between
two cavitation models one is the Schneer-Sauer model and the other is
the Zwart model with LES as a turbulent model. Finally, it concluded
that the Schneer-Sauer model is more accurate than the Zwart
cavitation mass transfer model for unsteady large-scale cavitation
shedding. And the 3D study gives more information about the unsteady
shedding along the span-wise direction. An important assumption made in this paper is a
homogenous flow for numerical study so-called bubbles two-phase
flow is based on the Rayleigh-Plessent equation and it neglects the
effects of surface tension and viscous damping along with condensation
and vaporization is controlled by barotropic state law. These
assumptions  are the basic assumption for the thesis
and the Schneer-Sauer model is chosen as a cavitation mass transfer
model to deal with a higher angle of attack in hydrofoil.  
  

In reference, \cite{Zhao2021} the work focus on the comparison of$k-\omega
$ sst with modified $k-\omega$ sst and LES Smagorinsky model on
NACA0012 hydrofoil at a higher angle of attack $8^{\circ}$ and fixed
cavitation number $\sigma$. They concluded that at a higher angle of
attack above $8^\circ$ it is better to use the LES model and
below this angle of attack is better to use $k-\omega$ sst turbulence
model. The filtered-based model of $k-\omega $ sst in comparison with
base $k-\omega$ sst does not show much difference in the result at an angle of attack $8^\circ$. The conclusion 
of this paper is taken into the thesis by using  $k-\omega$ sst as the base turbulence model for all
angles of attack which is well below $8^\circ$ along with constant
cavitation number $\sigma$. The boundary condition explicitly given in
this paper was used as the boundary condition for the case setup and
the experiment result given in this paper will help compare the numerical result of the different angles of attack with
fixed cavitation numbers.\\

Partial closed types cavity is our concern.
To understand partial closed and open type cavity, reference
\cite{ceccio2001} in which they conduct a study on unsteady cloud
shedding in NACA0009hydrofoil where they conclude that the re-entrant
jet was the week in the open cavity and there will be turbulent
reattachment. But in the closed partial cavity, there is an energetic
re-entrant jet with turbulence reattached. Here it shows that minimum
cavity thickness condition associated with cavity length is required
to observe the re-entrant jet and cloud shedding. This phenomenon was also explained in \cite{FundamentalsofCavitation.2004}. 

In  reference, \cite{Bensow2010} the paper give
an important idea about the need of using incompressible segregated
PISO algorithm. According to this paper, the pressure
equation needs some special attention to increase the numerical
stability along with the mass transfer model affecting the
velocity-pressure coupling while dealing with the incompressible implicit LES turbulence model. For this reason, it is better to neglect the effect of
compressibility and non-condensible gases. This also mentioned computational
difficulty in using the LES turbulence model as it requires large
computational power and cost. LES also requires a very fine mesh with a
very computational burden and it is very difficult to obtain a
grid-independent solution \cite{ZHANG2017} for 3D cases. But the thesis
 focused on 2D so it is better to use the $k-\omega $  sst model with the compressible segregated PIMPLE algorithm 
referred from \cite{Zhao2021}.  

The basic understanding of the
cavitation in hydrofoil and multiphase flow are studied and content
that supports writing the thesis is from reference textbook
\cite{FundamentalsofCavitation.2004, CavitationandBubbleDynamics.1995,
  brennen2005}and for introduction to turbulence from \cite{pope2000,ANSYS}, and
to understand about OpenFOAM, the guide released by OpenFOAM
foundation also used in this work\cite{OpenFoam}.\\ In thesis, the chapter1 will
cover all the basic understanding of the cavitation which is necessary
to know what is happening in the simulation. Chapter 2 will cover the
formulation required to run the simulation. It comprises of
formulation related to basic fluid dynamics and assumption-based fluid
flow equation and equation related to turbulence. Chapter 3
includes case setup comprises of geometry, mesh, boundary condition
required to run in OpenFOAM simulation.  Chapter 4 includes a result
discussion where the postprocessing data from OpenFOAM are tabulated
and compared with the experimental result. Finally, the conclusion
where results are concluded with the expert's suggestion who support me
to strengthen the work.


