\chapter*{Introduction}
\label{chap:introduction}
\renewcommand{\thepage}{\arabic{page}}
Cavitation is a very complex two-phase flow phenomenon. 
It can have multiple negative impacts on hydraulic machines 
such as vibration due to pressure fluctuation, high acoustic emission, 
solid surface erosion, and a drop-down in mechanical performance. Therefore 
it is necessary to investigate the physical mechanism for controlling cavitation
 in engineering applications. A primary objective of this study is to simulate 
 unsteady cloud shedding in cavitation and the re-entrant jet of a closed type 
 partial cavity over a NACA0012 hydrofoil at different angles of attack with a 
 fixed cavitation number. The cloud shedding phenomenon is reproduced by using 
 the $k- {\omega}$ sst as a turbulence model and homogeneous multiphase flow model. 
 The present study applies a mass transfer model developed by Schneer and Sauer called 
 the Schnerr Saucer model. In addition, the limitations of the $k-\omega$ sst turbulent 
 model at higher angles of attack will also be discussed.
 The solver InterFoam in the OpenFOAM\cite{OpenFOAM-9,Peric} source code is used to validate the 
 results by running a multiphase flow setup.\\ 
 
 The work in reference paper \cite{ZHANG2017} focused on a 3D Clark-Y hydrofoil by 
 utilizing an improved filter-based turbulence model based on density correction 
 to control the overprediction of turbulence viscosity. In contrast, without using the filter-based model, 
 the overprediction of turbulent viscosity causes the re-entrant to lose momentum. 
 As a result, the re-entrant jet was unable to cut the cavity interface, preventing cloud shed. The other influencing parameter is 
 the maximum density ratio which could affect the mass transfer 
 rate between the liquid and vapor. Due to the increase in the maximum density ratio, 
 the mass transfer rate between the liquid and vapor phases increases. In addition, 
 the cavity length, cavity depth, and thickness are significantly increased. 
 A large vapor cavity volume can result from mass transfer from liquid to vapor 
 at a high maximum density ratio. According to this article, two-equation turbulence 
 models cannot be used to simulate large-scale cavitation shedding due to turbulent 
 viscosity overprediction. The study concludes that the two-equation model does not 
 perform well near the wall of hydrofoil where viscous flow dominates but shows favorable results 
 away from the foil in the yplus outer region where inertia dominates. 
 ANSYS was used for the simulation.\\

According to reference papers \cite{JI2015, JI2017}, the simulations 
were conducted on the NACA66 hydrofoil, which explains the 3D effect 
of cavitation vortex interaction when cloud shedding occurs. The authors compare LES Wall
Adapting the local eddy-viscosity model over the LES WALE model which has
a better ability to reproduce the laminar to turbulent transition in
comparison with LES Smagorinsky. The cavitation vortex interaction
includes vortex stretching, vortex dilation (due to volumetric
expansion/contraction) baroclinic torque term (due to misaligned
pressure gradient and density gradient). The researchers concluded that, 
as cavitation grows, there is a strong vortex cavitation interaction in 
the shedding vapor cloud, and that the vortex stretching, dilation term 
is vital to the transition from 2D to 3D. As the shed vapor cloud collapses 
downstream and the attached cavity shrinks, the transition from 3D to 2D occurs. 
The attached cavitation and boundary layer becomes thin during this process. 
This transition will carry out after a regular interval. In comparing experimental pressure 
fluctuation to numerical pressure fluctuation, a shift in the result is 
due to the streamwise vortex (one reason) in the real flow setup, which 
cannot be added to the simulation. It is clear from this paper that the 
cavitation shedding condition is essential for producing cloud shedding 
in the simulation along with the 3D effect, and its contribution has an 
effect on the numerical result. The simulation was performed in ANSYS.\\
 


In reference \cite{Hidalgo2014,Hai2020,Hidalgo2014-2} the work
was validated in OpenFOAM on NACA66 hydrofoil. It compares two cavitation models, 
one being the Schneer-Sauer model and the other being the Zwart model with 
LES as a turbulent model. They concluded that the Schneer-Sauer model 
is more accurate than the Zwart cavitation mass transfer model for 
unsteady large-scale cavitation shedding. Additionally, the 3D study 
provides more detailed information about the unsteady shedding along the span-wise 
direction. This paper also assumes a homogeneous flow, so-called bubbles two-phase flow, 
for numerical analysis using the Rayleigh-Plessent equation, with no regard for 
surface tension and viscous damping. Another hypothesis made in this article is 
that condensation and vaporization are controlled by barotropic state law. Therefore, these ideas are the basis of the present work. 
As a result of these assumptions, the Schneer-Sauer model has been 
selected as a cavitation mass transfer model to account for a higher angle of attack on hydrofoils.\\

In reference \cite{Zhao2021} the work focuses on the comparison of $k-\omega $ sst 
 with modified  $k-\omega$ sst and LES Smagorinsky model on NACA0012 hydrofoil at a 
higher angle of attack $8^{\circ}$ with fixed cavitation number $\sigma$. The authors 
concluded that at an angle of attack greater than $8^\circ$ it is better to apply 
the LES model. On the other hand, if an angle of attack is less than $8^\circ$ is 
better to use the modified $k-\omega$ sst turbulence model. The filtered-based 
model of $k-\omega $ sst in comparison with base $k-\omega$ sst does not show much 
difference in the result at an angle of attack of $8^{\circ}$. This paper's conclusion 
is incorporated into the thesis by using the $k-\omega$ sst as the base turbulence model 
for any angle of attack less than $8^{\circ}$. This paper's boundary conditions are used 
to determine the case set up in the current work. To evaluate the numerical results 
obtained from the present work, we used experimental results derived from this paper.\\

Partially closed types of the cavity are our concern.  To understand partial
closed and open type cavity, reference \cite{ceccio2001} in which the authors
investigate unsteady cloud shedding in NACA0009 hydrofoil and
they concluded that the re-entrant jet was not energetic in the open cavity
but there was a turbulent reattachment. Alternatively the closed partial
cavity, there is an energetic re-entrant jet with turbulence
reattached. Here it shows that the minimum cavity thickness condition
associated with cavity length is required to observe the re-entrant
jet and cloud shedding. This phenomenon was also explained in
\cite{FundamentalsofCavitation.2004}.\\

In reference \cite{Bensow2010} the paper gives an important idea about
the need of using an incompressible segregated PISO algorithm. According
to this paper, the pressure equation needs some special attention to
increase the numerical stability along with the mass transfer model
affecting the velocity-pressure coupling while dealing with the
incompressible implicit LES turbulence model. For this reason, it is
better to neglect the effect of compressibility and non-condensable
gases. This paper also mentions the computational difficulty in using the LES
turbulence model as it requires large computational power and
cost. LES also requires a very fine mesh with a high computational
burden and it is very difficult to obtain a grid-independent solution
\cite{ZHANG2017} for 3D cases. But the thesis focused on 2D so it is
better to use the $k-\omega $ sst model with the compressible
segregated PIMPLE algorithm \cite{Zhao2021}.\\

The basic understanding of the cavitation in hydrofoil and multiphase
flow is studied and the content that supports this thesis is from
reference textbooks \cite{FundamentalsofCavitation.2004,
  CavitationandBubbleDynamics.1995, brennen2005}, for introduction
to turbulence refer to \cite{pope2000,ANSYS}, and to understand about
OpenFOAM, please consider the OpenFOAM guide reported in \cite{OpenFoam}.\\ In this document, the chapter1 will cover all the
basic understanding of cavitation which is necessary to know what
is happening in the simulation. Chapter 2 will cover the formulation
required to run the simulation. It comprises formulation related to
basic fluid dynamics and assumption-based fluid flow equations and
equations related to turbulence. Chapter 3 includes case setup, geometry, mesh, boundary condition which are required to run the
OpenFOAM simulation.  Chapter 4 includes a result discussion where the
postprocessing data from OpenFOAM are tabulated and compared with the
experimental results. Finally, the conclusion is reported in chapter 5 where the results are
concluded with the expert's suggestion who supports me to strengthen
the work.\\


